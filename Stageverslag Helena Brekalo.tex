\documentclass[10pt,a4paper]{article}
\usepackage[utf8]{inputenc}
\usepackage{amsmath}
\usepackage{amsfonts}
\usepackage{amssymb}
\usepackage{todonotes}

\begin{document}

\begin{titlepage}

\newcommand{\HRule}{\rule{\linewidth}{0.5mm}} % Defines a new command for the horizontal lines, change thickness here

\center % Center everything on the page
 
\textsc{\LARGE KU Leuven}\\[1.5cm] % Name of your university/college
\textsc{\Large 1 Ma Ingenieurswetenschappen: Computerwetenschappen}\\[0.5cm] % Major heading such as course name
\textsc{\large Bedrijfservaring: Computerwetenschappen}\\[0.5cm] % Minor heading such as course title


\HRule \\[0.4cm]
{ \huge \bfseries Stageverslag bedrijfservaring Alcatel-Lucent}\\[0.4cm]
\HRule \\[1.5cm]
 

\Large \emph{Author:}\\
Helena \textsc{Brekalo}\\[3cm]

{\large \today}\\[3cm] % Date

\vfill % Fill the rest of the page with whitespace

\end{titlepage}

Helena Brekalo\\
1e master Computerwetenschappen (Veilige software)\\
\medskip
\\
\textbf{Gegevens bedrijf:}\\
Alcatel-Lucent\\
Copernicuslaan 50\\
2018 Antwerpen\\
\medskip
\\
\textbf{Gegevens stagebegeleider:}\\
Bart Hemmeryckx-Deleersnijder\\
E-mail: bart.hemmeryckx-deleersnijder@alcatel-lucent.be\\
GSM-nummer: +32 472 95 00 65\\
Vast nummer binnen het bedrijf: +32 3 240 8525
\medskip
\\
\textbf{Stageperiode:}\\
1 juli - 7 augustus\\
7 september - 11 september
\clearpage

\begin{center}
\textbf{Abstract}

\end{center}
\tableofcontents
\clearpage

\section{Het bedrijf: Alcatel-Lucent}
Alcatel-Lucent is een Frans bedrijf met de hoofdzetel in Frankrijk en vestigingen in Amerika, Azi\"e, Europa (met dus onder andere een vestiging in Antwerpen), het Midden-Oosten \\todo{hoofdletters?} en Afrika. Alcatel kent een lange historie van overnames, met de meest recente overname die door Nokia, begin 2015.\\
De eerste funderingen van Alcatel werden gelegd in 1869, met de opstart van Gray and Barton in Cleveland, Ohio. Een tien jaar later worden ze overgenomen door American Bell, wat in 1925 leidt tot het ontstaan van Bell Telephone Laboratories. In deze "Bell Labs" worden verschillende grote ontwikkelingen verwezenlijkt, waaronder de eerste lange afstandstelevisietransmissie en de uitvinding van de batterij op zonne-energie. \todo{beter schrijven!} In 1984 worden ze overgenomen door C\^ables de Lyon en worden ze een Frans bedrijf. In 2006 mergen Alcatel en het Amerikaanse Lucent Technologies, waarop de naam verandert naar Alcatel-Lucent. In 2015 werd Alcatel-Lucent overgenomen door Nokia, wat effectief zal worden in april 2016. Door deze overname zal Nokia het tweede grootste bedrijf in wireless zijn, op Ericsson na. Op deze manier wordt de concurrentie met het Chinese Huawei vergroot op het gebied van draadloze communicatietechnologie.\\
Alcatel-Lucent heeft verschillende afdelingen, zijnde het \textit{core networking segment}, wat IP routing, transport en platforms omvat en het \textit{access segment}, wat wireless en fixed access omvat, alsook licensing en \textit{managed services}. In Antwerpen wordt er aan al deze segmenten gewerkt; zelf werd ik tewerkgesteld in de fixed access afdeling, namelijk Motive Network Analyzer-Copper (Motive NA-C). Zij staan ervoor in om het gebruik van het kopernetwerk te optimaliseren en fouten te detecteren en verhelpen.\\ \todo{nog tekst toevoegen?}


\section{Stage}
\subsection{Afdeling: Motive NA}
De stage ging door op de afdeling Motive Network Analyzer - Copper (NA-C) bij het team dat ook onder begeleiding staat van mijn stagebegeleider Bart Hemmeryckx-Deleersnijder. Motive NA-C heeft een team in Antwerpen, Chennai en Bangalor, waarbij de development voornamelijk in Antwerpen gebeurt en het testen in Chennai en Bangalor. Er zijn altijd ook enkele testers van Chennai en Bangalor in Antwerpen, om zo de communicatie tussen de teams te verbeteren en de samenwerking te vergemakkelijken.
\subsection{Werkmethoden}
\subsubsection{Scrum}
Het team van Bart werkt volgens de scrum-methode, zoals kort was aangehaald tijdens de hoorcollege's van Software-Ontwerp. Scrum is een voorbeeld van een agile werkmethode, waarbij men ervan uitgaat dat de vereisten voor het project kunnen en zullen veranderen gedurende het project. Bij het klassieke watervalmodel stelt men eerst de vereisten op, om deze vervolgens te implementeren, dit te evalueren en dan te onderhouden. Het probleem hierbij is dat de communicatie tussen de klant en het bedrijf 
niet optimaal verloopt, gezien de klant nog van gedachte kan veranderen over de requirements die hij wil. Als deze pas in de maintenance fase ontdekt en aangekaart worden, is het heel kostelijk om (grote) veranderingen opnieuw door te voeren.\\
Bij een agile werkmethode, waar scrum een onderdeel van is, gaat men er van uit dat de klant op voorhand niet perfect kan weten wat hij wil en dat hij dus de requirements zal aanpassen.\\
De term "scrum" is afkomstig vanuit rugby, waarbij de twee teams voorovergebogen tegen elkaar leunen, met de hoofden bij elkaar. De bal wordt dan in deze groep gegooid, waarop ze proberen om de bal als eerste in hun bezit te krijgen.


Men werkt in \textit{sprints} van twee weken, waarbij er een tweewekelijks doel wordt gesteld dat wordt ge\"evalueerd en eventueel aangepast.

-scrum
-agile
-> jenkins: continuous integration

-zelf ook: jira en conlfuence -> link ernaartoe!
-erg betrokken: standup meetings, sprint meeting, planning meeting, demo (Zelf een moeten geven)
-behulpzaam team (onderling, ook voor mij)
\subsection{Stage-opdracht}
\subsection{Resultaat}

\section{Relatie stage en opleiding}
-scrum: swop
-pair programming: swop
-zelf project plannen en uitwerken, met tussentijdse evaluaties
\section{Krititsche reflectie competenties}

\section{Conclusie}

\section{Annex}

\end{document}